\chapter{Tổng Kết} 
\label{chap:conclusion}

\begin{chapabstract}
Chương \ref{chap:conclusion} trình bày các kết quả của khóa luận, bao gồm những gì chúng tôi đã học và đạt được thông qua các thực nghiệm. Chương này kết thúc với đề xuất của chúng tôi cho công việc nghiên cứu trong tương lai.
\end{chapabstract}

\section{Kết quả đạt được}

Trong quá trình thực hiện luận án này, chúng tôi đã dành rất nhiều thời gian nghiên cứu về Phát hiện mã độc và Học máy, bao gồm các mạng Neural Networks và Gradient Boosting.
Chúng tôi đã học được và phân biệt cách phát hiện phần mềm độc hại tĩnh và động.
Chúng tôi cũng đã phân tích chi tiết kỹ thuật và ý nghĩa của các tính năng được sử dụng trong các phương pháp phát hiện phần mềm độc hại.
Việc hiểu chúng là quan trọng cho cả việc hiểu các phương pháp state-of-the-art lẫn việc xây dựng và tối ưu hóa các tham số.

Chúng tôi trình bày và tối ưu hóa một phương pháp phát hiện phần mềm độc hại tĩnh sử dụng các tính năng được tạo thủ công bắt nguồn từ việc phân tích cú pháp các tệp PE và thuật toán Gradient Boosting Decision Trees (GBDT), một thuật toán học máy mạnh mẽ được sử dụng phổ biến trong thời gian gần đây.
Chúng tôi cố gắng để giảm thời gian đào tạo bằng cách giảm kích thước tính năng một cách thích hợp.
Thay vì sử dụng các tệp nhị phân thô, phương pháp được đề xuất của chúng tôi sử dụng các thống kê đơn gian để giảm mối quan tâm về quyền riêng tư của các tệp lành tính khác nhau và giúp dễ dàng yêu cầu tập dữ liệu cân bằng.

Kết quả thực nghiệm cho thấy phương pháp đề xuất của chúng tôi có thể đạt tới 99,394\% tỷ lệ phát hiện ở tỷ lệ báo động giả 1\% và với tỉ lệ báo động sai dưới 0,1\%, mô hình đạt tỷ lệ phát hiện 97,572\%, dựa trên hơn 600.000 mẫu đào tạo và 200.000 mẫu thử từ bộ dữ liệu EMBER \cite{anderson2018ember}.

\section{Hướng phát triển}

Nghiên cứu được tiến hành trong dự án này là một minh chứng khái niệm (proof-of-concept), chúng tôi có thể xác định một số hướng phát triển trong tương lai:

\begin{enumerate}
    \item \textbf{Giảm không gian đặc trưng. } Có thể giảm kích thước của vectơ đặc trưng. Các vectơ đầu vào có kích thước nhỏ hơn có thể làm tăng độ chính xác cho mô hình và mất ít thời gian đào tạo hơn.
    \item \textbf{Sử dụng những bộ dữ liệu khác. } 
    Mặc dù tập dữ liệu EMBER rộng, bao gồm hầu hết các loài phần mềm độc hại, nhưng nó không bao gồm tất cả các loại có thể. Thu thập tập dữ liệu là một tác vụ đòi hỏi nhiều thời gian và công sức, đặc biệt là trong lĩnh vực phát hiện phần mềm độc hại. Với việc sử dụng các tính năng không có định dạng, chúng tôi có thể nhận được nhiều mẫu hơn từ các tổ chức bảo mật trong tương lai.
    \item \textbf{Triển khai phương pháp tiếp cận trong máy tính cục bộ. } Chúng tôi đã cố gắng triển khai mô hình với định dạng ONNX và nền tảng Windows ML nhưng không thành công vì phiên bản thử ngfhieejm của Windows ML thay đổi nhanh chóng và có nhiều giới hạn. Chúng tôi dự định xây dựng một ứng dụng trình diễn để cho thấy rằng các bộ công cụ phát hiện phần mềm độc hại dựa trên học máy có thể chạy tốt trong máy tính cá nhân.
\end{enumerate}

