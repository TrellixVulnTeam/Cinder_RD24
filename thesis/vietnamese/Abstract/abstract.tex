% ************************** Thesis Abstract *****************************
% Use `abstract' as an option in the document class to print only the titlepage and the abstract.
\renewcommand{\abstractname}{Executive Summary}

\begin{abstract}
Phát hiện mã độc tĩnh là một lớp cần thiết trong các hệ thống bảo mật, nó cố gắng phân loại các mẫu là độc hại hoặc lành tính trước khi thực thi chúng.
Tuy nhiên, hầu hết các nghiên cứu liên quan gặp các vấn đề về khả năng mở rộng, ví dụ, các phương pháp sử dụng neural networks thường mất rất nhiều thời gian đào tạo \cite{raff2017malware} hoặc sử dụng các tập dữ liệu mất cân bằng \cite{saxe2015deep,vu2017metamorphic}, làm cho các số liệu đánh giá bị gây hiểu nhầm trong thực tế.

Trong nghiên cứu này, chúng tôi tìm hiểu về cả hai phương pháp nhận diện mã độc trên hệ điều hành Windows: phát hiện mã độc tính (static malware detection) và phát hiện mã độc động (dynamic malware detection). 
Chúng tôi tiến hành các thí nghiệm để áp dụng các phương pháp học máy trong phát hiện mã độc tĩnh. 
Hơn nữa, chúng tôi đề xuất phương pháp phát hiện phần mềm độc hại tĩnh sử dụng Portable Executable analysis và thuật toán học máy Gradient Boosting Decision Tree. 
Chúng tôi tập trung vào việc giảm thời gian đào tạo bằng cách giảm số lượng tính năng một cách thích hợp.
Các kết quả thử nghiệm cho thấy phương pháp đề xuất của chúng tôi có thể đạt được tỷ lệ phát hiện lên tới 99,394 \% với tỷ lệ cảnh báo sai là 1\%, và khi giới hạn tỷ lệ báo động giả dưới 0,1\% thì tỷ lệ phát hiện là 97,572 \%, dựa trên hơn 600.000 mẫu đào tạo và 200.000 mẫu kiểm thử từ bộ dữ liệu Endgame Malware BEnchmark for Research (EMBER) \cite{anderson2018ember}.
\end{abstract}
