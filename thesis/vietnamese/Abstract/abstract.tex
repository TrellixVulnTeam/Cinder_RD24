% ************************** Thesis Abstract *****************************
% Use `abstract' as an option in the document class to print only the titlepage and the abstract.
\renewcommand{\abstractname}{Executive Summary}

\begin{abstract}
Static malware detection is an essential layer in a security suite, which attempts to classify samples as malicious or benign before execution. 
However, most of the related works incur the scalability issues, for examples, methods using neural networks usually take a lot of training time \cite{raff2017malware}, or use imbalanced datasets \cite{saxe2015deep,vu2017metamorphic}, which makes validation metrics misleading in reality.

In this study, we research the two essential approaches for malware detection (i.e., static and dynamic analysis) and conduct experiments to apply the machine learning methods in static malware detection. 
Furthermore, we propose a static malware detection method by Portable Executable analysis and Gradient Boosting Decision Tree algorithm. 
We manage to reduce the training time by appropriately reducing the feature dimension. 
The experiment results show that our proposed method can achieve up to 99.394\% detection rate at 1\% false alarm rate, and score results in less than 0.1\% false alarm rate at a detection rate 97.572\%, based on more than 600,000 training and 200,000 testing samples from Endgame Malware BEnchmark for Research (EMBER) dataset \cite{anderson2018ember}.
\end{abstract}
